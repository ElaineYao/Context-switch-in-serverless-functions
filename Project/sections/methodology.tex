


\subsection{Challenges}
Measuring the context switch time in serverless functions has several challenges:
\begin{itemize}
	\item [C1] Characteristics of context switch in serverless environments
	Context switch is triggered differently in serverless functions compared to in traditional operating systems as it's scheduled by the cloud provider.
	And the service provider has the tendency to trigger more functions in the same hardware to get more profits. 
	Thus, context switch may happen not only due to performance factors like multi-threading/processing, but also provider's profit considerations.
	\item [C2] Benchmark accuracy
	As no such benchmark is provided so far, we have to analyze the factors in our own benchmark that may lead to the potential variation to golden value, 
	in order to reason about the accuracy of the measurement.
	% Variations to golden values in measurement
	\item [C3] 
\end{itemize}

\subsection{}
	To tackle \emph{C1}, we design the following experiments to analyze the factors influencing context switch in serverless environments.
	We measure the execution time elapse between two adjacent lines of code repeatedly under different configurations in serverless functions.
	Shahrad \emph{et al.} \cite{serverless-main} shows that the execution time in serverless functions is influenced by function invocation frequency, memory size and function type.
	Thus these configurations will include the function invocation rate, memory allocated.
	Theoriotically, the measured time elapse should be 0 as there is not execution between these two lines. 
	However, our initial experiments show that this is neither 0 nor a fixed value, indicating that interruption happens between these two lines
	and the time elapse changes with different configurations. 
	Although the interruption is not limited to context switch but also page faults handling or container management, 
	it can still provides insights in the influence of different configurations on context switch time in serverless environments.


\subsection{Combine various benchmarks}
	To tackle \emph{C2}, we first run our own designed benchmarks on local Linux system and compare the results with published benchmarks\cite{cs-lmbench,cs-pipes,cs-arm,cs-web},
	 ensuring the correctness of the proposed ones. Then we deploy the new benchmarks in serverless functions and compare the results. 
	 If one of the benchmarks is consistently higher or lower than others, then we'll explore the reasons and this can help us improve our accuracy.

	The previous benchmarks in context switching on Linux system mainly consider %some fa
	Our initial study to \emph{C1} finds that context switch in serverless environments will also be influenced by invocation rate and memory allocation.   

	Currently, the new benchmarks are designed as the prior benchmarks with different configurations.
	The prior benchmarks we are choosing are:
	\begin{enumerate}
		\item 
	\end{enumerate}
	Next, we'll implement these functions with different invocation rate and memory allocation setting in the cloud.

% "Based on these traditional methods, we also used pipe communication to implement frequent context switches between two processes."
